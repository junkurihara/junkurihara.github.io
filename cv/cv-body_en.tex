\section*{Education}

\begin{itemize}
  \item \textbf{Tokyo Institute of Technology}\\
	\hspace*{4ex}\textit{Ph.D. in Engineering}, September 2012\\
	\hspace*{4ex}Dissertation: A Study on Design and Security Analysis of Secret Sharing Schemes\\
	\hspace*{4ex}Supervisor: Professor Tomohiko Uyematsu

  \item \textbf{Tokyo Institute of Technology}\\
	\hspace*{4ex}\textit{M.E. in Communications and Integrated
	Systems}, March 2006\\
	\hspace*{4ex}Thesis: Nonbinary Coding Systems Approaching the Shannon Limit by Using Product Accumulate Codes\\
	\hspace*{4ex}Supervisor: Professor Tomohiko Uyematsu

  \item \textbf{Tokyo Institute of Technology}\\
	\hspace*{4ex}\textit{B.E. in Computer Science}, March 2004\\
	\hspace*{4ex}Thesis: Software Radio Receiver Utilizing RF Filter Bank\\
	\hspace*{4ex}Supervisor: Professor Hiroshi Suzuki
\end{itemize}

%\newpage

\section*{Work Experience}
\textbf{Associate Professor}\\
\hspace*{4ex}\textit{Graduate School of Applied Informatics, University of Hyogo, Japan}\\
\hspace*{4ex}2020 Jan.--Present

\hspace*{4ex} His responsibility includes researching on the wide range of security and dependability in networking and computing, and also lecturing mathematics, security and networking.
Currently he is mainly working on the following research and development projects.
\begin{itemize}
\item \hspace*{4ex} \underline{Security, Privacy and Anonymity in Domain Name Systgem (DNS)}\\[0.5ex]
\hspace*{6ex}\begin{minipage}{0.9\linewidth}
He developed and leads an experimental software project of anonymized DNS protocols called \textit{Mutualized oblivious DNS} ($\mu$ODNS) by fully leveraging his networking and cryptography R\&D background. Extending IETF draft technologies, implementing its proof-of-concept software as an open source software, and operating developed DNS servers on the Internet. (See GitHub repos for codes)
\end{minipage}
\item \hspace*{4ex} \underline{Private Information Retrieval}\\[0.5ex]
\hspace*{6ex}\begin{minipage}{0.9\linewidth}
He researches and develops coding-theoretic protocols of \emph{private information retrieval} to hide requester's interests from the server's observation and preserve the privacy of the requester. He especially focuses on the dependability of the protocol against the active attacks, i.e., destruction of the information.
\end{minipage}
\end{itemize}
\vspace*{2ex}

\textbf{Principal Researcher/Software Engineer}\\
\hspace*{4ex}\textit{Zettant Inc., Japan}\\
\hspace*{4ex}2018 Jan.--Present

\hspace*{4ex} Mainly working on research and development projects related to cryptographic primitive library, blockchain architecture, security and access control systems. In particular, he launched the project of security platform, called \emph{SecurityHub}, enabling the easy-to-use and secure usage of crypto keys. In the project, he presented the initial concept of SecurityHub, designed its initial and detailed architectures, and he has already submitted some patents. Currently he is leading its development to release the platform as a service shortly.
He is also committing an open-source software project called \emph{jscu} that is a TypeScript cryptographic library providing unified APIs for browsers and Node.js.
\vspace*{2ex}

\textbf{Visiting/Cooperate Scholar}\\
\hspace*{4ex}\textit{Advanced Telecommunication Research Institute International (ATR), Japan}\\
\hspace*{4ex}2019 Jun.--2019 Dec. (Cooperate Researcher), and 2020 May--Present (Visiting Scholar).

\hspace*{4ex} Working on research and development projects related to security in networking architectures including ICN as a researcher.
\vspace*{2ex}



\textbf{Strategic Planner, Engineer}\\
\hspace*{4ex}\textit{KDDI Corp., Japan}\\
\hspace*{4ex}2016 Oct.--2017 Dec.

\hspace*{4ex} His responsibility mainly included the followings related to the mobile core network architecture:
\begin{itemize}
 \item \hspace*{4ex}\begin{minipage}{0.9\linewidth}
- Planning the mobile core network architecture and its road-map for the future (e.g., LPWA, 5G, etc.) mobile services.\end{minipage}
 \item \hspace*{4ex}\begin{minipage}{0.9\linewidth}
- Designing the mobile core network structure (e.g, 4G Evolved Packet Core) and its platform structure (e.g., charging system, networking functions behind the PGW, etc.) for the various business demands.
\end{minipage}
 \item \hspace*{4ex}\begin{minipage}{0.9\linewidth}
- Planning (negotiating) the strategic collaboration with global network operators and hardware/software manufactures in the various technological area (e.g., LPWA, 5G networking use cases, etc.).
\end{minipage}
\end{itemize}
\vspace*{2ex}

% \begin{itemize}
%  \item  \hspace*{4ex}
% \underline{xxx},
% 2016--Present\\[0.5ex]
% \hspace*{6ex}
% \begin{minipage}{0.9\linewidth}
% xxx
% \end{minipage}
% \end{itemize}

%\begin{itemize}
%\item
\textbf{Researcher}\\
\hspace*{4ex}\textit{KDDI R\&D Labs.\@, Inc., Japan}\\
\hspace*{4ex}2006 Apr.--2016 Sep.

\hspace*{4ex}Mainly worked on the following research projects for the security and networking architecture.


\begin{itemize}
\item \hspace*{4ex}
\underline{Research project on information centric networking (ICN) architecture and its security},
2013--2016\\[0.5ex]
\hspace*{6ex}
\begin{minipage}{0.9\linewidth}
Started this project in order to design and lead the clean-slate ICN architecture in KDDI core network.
First created a new access control framework during the one-year stay in Palo Alto Research Center (PARC), CA, USA as a visiting researcher. Next, struggled with the reduction of router's workload and created a new technology to realize a dramatically-lightweight processing using a grouping method of request messages. Thirdly, as a member of ICN2020 project, launched a new sub-project on a novel ICN-specific method for censorship circumvention, which maximize the benefit of ICN like in-network caching.
\end{minipage}

\item \hspace*{4ex}
\underline{Research project on secure and reliable network coding/distributed storage},
2011--2015\\[0.5ex]
\hspace*{6ex}
\begin{minipage}{0.9\linewidth}
Launched this project in order to realize efficient and reliable communication
 in the network of the future.
First proposed an explicit construction of
 universal strongly secure network coding scheme using maximum rank
 distance codes, which had been remained an open question.
Next, pioneered the theory of new code parameters generalizing the rank distance, and revealed
that these parameters precisely characterize the security and
error-correction capability of universal secure network coding
scheme.
\end{minipage}

\item \hspace*{4ex}
\underline{Research project on secret sharing schemes and linear error-correcting codes},
2010--2014\\[0.5ex]
\hspace*{6ex}
\begin{minipage}{0.9\linewidth}
Launched this project in order to design secret sharing schemes suitable
 for cryptographic applications.
Revealed that the security performance of secret sharing schemes based
on linear codes is precisely expressed in terms of parameters of
 the codes, which are called relative dimension/length profile and
 relative generalized Hamming weight. Further, demonstrated that security analysis in existing
 researches by the minimum Hamming weight are loose and not precise.
\end{minipage}

\item \hspace*{4ex}
\underline{Development of authentication method for broadcasting stream},
2007--2009\\[0.5ex]
\hspace*{6ex}
\begin{minipage}{0.9\linewidth}
Proposed a new authentication method for TV broadcasting stream, which is suitable for
resource constraint environments. Developed mobile terminals with the
 method, and demonstrated its efficiency and effectiveness.
The mobile terminals which the scheme have been used at the
 demonstration in the 34th G8 summit took place in Toyako, Hokkaido, Japan.
\end{minipage}

\item \hspace*{4ex}
\underline{Design of secret sharing schemes and their applications},
2006--2012\\[0.5ex]
\hspace*{6ex}
\begin{minipage}{0.9\linewidth}
Pioneered this area of high-speed secret sharing schemes. Proposed a
 novel construction of a secret sharing scheme realizing extremely rapid
 computations, which uses only exclusive-or operations to encode and
 decode the secret data.
Currently, this novel scheme is used in several commercial services of
 KDDI and other companies as a core technology,
 e.\@g.\@, a secure distributed file system, a backup
 service using multiple cloud storage services, etc.
The core library for the scheme itself is released as a product called ``SProDa (secure protection of data)'' from KDDI R\&D Labs., Inc. (See \href{http://www.kddilabs.jp/english/products/sproda.html}{http://www.kddilabs.jp/english/products/sproda.html}.)
\end{minipage}
\end{itemize}

\textbf{Visiting Researcher}\\
\hspace*{4ex}\textit{Palo Alto Research Center, CA, USA}\\
\hspace*{4ex} 2013 Sep.--2014 Sep.


\begin{itemize}
\item \hspace*{4ex}
\underline{Research project on access control in information centric networking (ICN)},
2013--2014\\[0.5ex]
\hspace*{6ex}
\begin{minipage}{0.9\linewidth}
Launched this project in order to design a ICN-specific framework for access control. Designed the framework using a new network message called manifest, which flexibly realizes arbitrary access control instances.
\end{minipage}

\end{itemize}
%\end{itemize}

\section*{Publications}

\subsection*{Peer-Reviewed Journal Articles and Letters}
\begin{enumerate}
 \item I.~Kurihara, \underline{J.~Kurihara} and T.~Tanaka, ``A New Security Measure in Secret Sharing Schemes and Secure Network Coding,'' accepted for publication in \textit{IEEE Access}, 2024.
 \item \underline{J.~Kurihara}, T.~Kubo and T.~Tanaka, ``$\mu$ODNS: A Distributed Approach to DNS Anonymization with Collusion Resistance,'' \textit{Computer Networks}, Elsevier, vol. 237, p. 110078, Dec. 2023.
 \item \underline{J.~Kurihara}, T.~Nakamura and R.~Watanabe, ``Private information retrieval from coded storage in the presence of omniscient and limited-knowledge Byzantine adversaries'', \textit{IEICE Transactions on Fundamentals of Electronics, Communications and Computer Sciences}, vol.~E104-A, no.~9, pp.~1271--1283, Sep. 2021.
 \item Y.~Koike, T.~Hayashi, \underline{J.~Kurihara} and T.~Isobe, ``Virtual Vault: A practical leakage resilient scheme using space-hard ciphers,'' \textit{IEICE Transactions on Fundamentals of Electronics, Communications and Computer Sciences}, vol.~E104-A, no.~1, pp.~182--189, Jan. 2021.
 \item \underline{J.~Kurihara}, and T.~Nakamura, ``On the resistance to Byzantine and unresponsive servers in code-based PIR schemes,'' \textit{IEICE Communications Express}, vol.~9, no.~7, pp.~342--347, Jul.~2020.
 \item K.~Ueda, K.~Yokota, \underline{J.~Kurihara}, A.~Tagami, ``Two-level named packet forwarding for enhancing the performance of virtualized ICN router,'' \textit{IEICE Transactions on Communications}, vol.~E102-B, no.~2, pp.~1813--1821, Sep.~2019.
 \item \underline{J.~Kurihara}, K.~Yokota, and A.~Tagami, ``List interest: Simply packing interests dramatically reduces router workload in content-centric networking,'' \textit{IEICE Transactions on Communications}, vol.~E99-B, no.~12, pp.~2520--2531, Dec.~2016.
 \item \underline{J.~Kurihara}, R.~Matsumoto, and T.~Uyematsu, ``Relative generalized rank weight of linear codes and its applications to network coding'' \textit{IEEE Transactions on Information Theory}, vol.~61, no.~7, pp.~3912--3936, Jul. 2015.
 \item \underline{J.~Kurihara}, and Y.~Miyake, ``Securing distributed storage systems based on arbitrary regenerating codes,'' \textit{IEICE Communications Express}, vol.~2, no.~10, pp.442--446, Oct. 2013.
 \item \underline{J.~Kurihara}, T.~Uyematsu, and R.~Matsumoto,
       ``Secret sharing schemes based on linear codes can be precisely
       characterized by the relative generalized Hamming weight,''
       \textit{IEICE Transactions on Fundamentals of Electronics,
       Communications and Computer Sciences},
       vol.~E95-A, no.~11, pp.~2067--2075, Nov. 2012.
 \item Y.~Nakano, \underline{J.~Kurihara}, S.~Kiyomoto, and T.~Tanaka, ``Stream cipher-based hash function and its security,'' \textit{Revised Selected Papers in the 7th International Joint Conference e-Business and Telecommunications, ICETE 2010, Athens, Greece, July 26--28, 2010}, ser.~Communications in Computer and Information Science, vol.~222, \hskip 1em plus 0.5em minus 0.4em\relax Heidelberg, Germany: Springer-Verlag, pp.~188--202, 2012.
 \item \underline{J.~Kurihara}, and T.~Uyematsu, ``A novel realization
       of threshold schemes over binary field extensions,''
       \textit{IEICE Transactions on Fundamentals of Electronics, Communications and Computer Sciences},
       vol.~E94-A, no.~6, pp.~1375--1380, Jun. 2011.
 \item \underline{J.~Kurihara}, S.~Kiyomoto, R.~Watanabe, and
       T.~Tanaka, ``A stream authentication method for one-seg
       broadcasting,'' \textit{Journal of the Institute of Image
       Information and Television Engineers},
       vol.~64, no.~12, pp.~1921--1932, Dec. 2010. (in Japanese)
 \item \underline{J.~Kurihara}, S.~Kiyomoto, K.~Fukushima, and
       T.~Tanaka, ``A fast $(k,L,n)$-threshold ramp secret sharing
       scheme,'' \textit{IEICE Transactions on Fundamentals of Electronics, Communications and Computer Sciences},
       vol.~E92-A, no.~8, pp.~1808--1821, Aug. 2009.
 \item \underline{J.~Kurihara}, S.~Kiyomoto, K.~Fukushima, and
       T.~Tanaka, ``On a fast $(k,n)$-threshold secret sharing scheme,''
       \textit{IEICE Transactions on Fundamentals of Electronics, Communications and Computer Sciences},
       vol.~E91-A, no.~9, pp.~2365--2378, Sep. 2008.
 \item \underline{J.~Kurihara}, S.~Kiyomoto, K.~Fukushima, and
       T.~Tanaka, ``A fast $(3,n)$-threshold secret sharing scheme using
       exclusive-or operations,''
       \textit{IEICE Transactions on Fundamentals of Electronics, Communications and Computer Sciences},
       vol.~E91-A, no.~1, pp.~127--138, Jan. 2008.
 \item A.~Deininger, S~ Kiyomoto, \underline{J.~Kurihara}, and
       T.~Tanaka, ``Security vulnerabilities and solutions in mobile
       WiMAX,'' \textit{IJCSNS International Journal of Computer Science
       and Network Security}, vol.~7, no.~11, pp.~7--15, Nov. 2007.
\end{enumerate}



\subsection*{Peer-Reviewed Conference Proceedings}
\begin{enumerate}
 \item R.~Watanabe, A.~Kubota, \underline{J.~Kurihara}, and K.~Sakurai, ``Extension of Resource Authorization Method with SSI in Edge Computing,'' to Appear in Proc. AINA 2024, Apr. 2024.
 \item R.~Watanabe, A.~Kubota, and \underline{J.~Kurihara}, ``Application of Generalized Deduplication Techniques in Edge Computing Environments,''  in \textit{Advanced Information Networking and Applications, Proceedings of the 37-th International Conference on Advanced Information Networking and Applications (AINA 2023), Volume 3}, ser. Lecture Notes in Networks and Systems, L. Barolli, Ed., vol.~655.\hskip 1em plus 0.5em minus 0.4em\relax Springer, Cham, 2023, pp.~585--596.
 \item R.~Watanabe, A.~Kubota, and \underline{J.~Kurihara}, ``Resource Authorization Methods for Edge Computing,''  in \textit{Advanced Information Networking and Applications, Proceedings of the 36-th International Conference on Advanced Information Networking and Applications (AINA 2022)}, ser. Lecture Notes in Networks and Systems, L. Barolli, F. Hussain, and T. Enokido, Eds., vol.~449.\hskip 1em plus 0.5em minus 0.4em\relax Springer, Cham, 2022, pp.~167--179.
 \item K.~Suksomboon, A.~Tagami, A.~Basu, and \underline{J.~Kurihara}, ``In-device proxy re-encryption service for information-centric networking access control,'' in \textit{Proceedings of the 43rd IEEE Conference on Local Computer Networks (LCN 2018)}, Chicago, IL, USA, Oct.~1--4, 2018, pp.~303--306.
 \item K.~Suksomboon, A.~Tagami, A.~Basu, and \underline{J.~Kurihara}, ``IPRES: In-device proxy re-encryption service for secure ICN,'' in \textit{Proceedings of the 4th ACM Conference on Information-Centric Networking (ICN 2017)}, Berlin, Germany, Sep.~26--28, 2017, pp.~176--177.
 \item K.~Ueda, K.~Yokota, \underline{J.~Kurihara}, and A.~Tagami, ``Towards the NFVI-assisted ICN: Integrating ICN forwarding into the virtualization infrastructure,'' in \textit{Proceedings of the 2016 IEEE Global Communications Conference (GLOBECOM 2016)}, Washington, DC, USA, Dec.~4--8, 2016.
 \item \underline{J.~Kurihara}, K.~Yokota, and A.~Tagami, ``A consumer-driven access control approach to censorship circumvention in content-centric networking,'' in \textit{Proceedings of the 3rd ACM Conference on Information-Centric Networking (ICN 2016)}, Kyoto, Japan, Sep.~26--28, 2016, pp.~186--194.
 \item K.~Yokota, K.~Sugiyama, \underline{J.~Kurihara}, and A.~Tagami, ``RTT-based caching policies to improve user-centric performance in CCN,'' in \textit{Proceedings of the 2016 IEEE 30th International Conference on Advanced Information Networking and Applications (AINA 2016)}, Crans-Montana, Switzerland, Mar.~23--25, 2016, pp.~124--131.
 \item \underline{J.~Kurihara}, K.~Yokota, K.~Ueda, and A.~Tagami, ``List interest: Packing interests for reduction of router workload in CCN 1.0,'' in \textit{Proceedings of IEEE MASS 2015 Workshop on Content-Centric Networking (CCN 2015)}, Dallas, TX, USA, Oct.~19--22, 2015, pp.~500--505.
 \item K.~Ueda, K.~Yokota, \underline{J.~Kurihara}, and A.~Tagami, ``A performance analysis of end-to-end fragmentation in content-centric networking,'' in \textit{Proceedings of IEEE MASS 2015 Workshop on Content-Centric Networking (CCN 2015)}, Dallas, TX, USA, Oct.~19--22, 2015, pp.~531--536.
 \item \underline{J.~Kurihara}, E.~Uzun, and C.~A.~Wood, ``An encryption-based access control framework for content-centric networking,'' in \textit{Proceedings of IFIP Networking Conference 2015}, Toulouse, France, May~20--22, 2015, pp.~1--9.
 \item \underline{J.~Kurihara}, T.~Uyematsu, and R.~Matsumoto, ``New
       parameters of linear codes expressing security performance of
       universal secure network coding,''
       in \textit{Proceedings of the 50th Annual Allerton
       Conference on Communication, Control, and Computing (Allerton
       2012)}, Monticello, IL, USA, Oct.~1--5, 2012.
 \item \underline{J.~Kurihara}, T.~Uyematsu, and R.~Matsumoto,
       ``Explicit construction of universal strongly secure network
       coding via MRD codes,''
       in \textit{Proceedings of 2012 IEEE International Conference on
       Information Theory (ISIT 2012)}, Cambridge, MA, USA, Jul.~1--6,
       2012, pp.~1483--1487.
 \item \underline{J.~Kurihara}, and T.~Uyematsu, ``Strongly-secure secret
       sharing based on linear codes can be characterized by generalized
       Hamming weight,'' in \textit{Proceedings of the 49th Annual Allerton
       Conference on Communication, Control, and Computing (Allerton
       2011)}, Monticello, IL, USA, Sep.~28--30, 2011, pp.~951--957.
 \item \underline{J.~Kurihara}, and T.~Uyematsu, ``Vulnerability of
       MRD-code-based universal secure error-correcting network codes
       under time-varying jamming links,'' in \textit{Proceedings of
       the Fourth International Conference on Communication Theory,
       Reliability, and Quality of Service (CTRQ 2011)}, Budapest,
       Hungary, Apr.~17--22, 2011, pp.~35--39.
 \item Y.~Nakano, \underline{J.~Kurihara}, S.~Kiyomoto, and T.~Tanaka, ``On
       a construction of stream-cipher-based hash functions,'' in
       \textit{Proceedings of SECRYPT 2010}, Athens, Greece,
       Jul. 26--28, 2010, pp.~334--343.
 \item C.~Cid, S.~Kiyomoto, and \underline{J.~Kurihara}, ``The Rakaposhi
       stream cipher,'' in \textit{Information and Communications
       Security, 11th International Conference, ICICS 2009, Beijing,
       China, December 14--17, 2009. Proceedings}, ser. Lecture Notes in Computer
       Science, S.~Qing, C.~J.~Mitchell and G.~Wang, Eds.,
       vol.~5222.\hskip 1em plus 0.5em minus 0.4em\relax Heidelberg,
       Germany: Springer-Verlag, 2009, pp.~32--46.
 \item \underline{J.~Kurihara}, S.~Kiyomoto, K.~Fukushima, and
       T.~Tanaka,
       ``A new $(k,n)$-threshold secret sharing scheme and its
       extension,'' in \textit{Information Security, 11th International
       Conference, ISC 2008, Taipei, Taiwan, September 15--18,
       2008. Proceedings}, ser. Lecture Notes in Computer
       Science, T.-C. Wu, C.-L. Lei, V. Rijmen and D.-T. Lee, Eds.,
       vol.~5222.\hskip 1em plus 0.5em minus 0.4em\relax Heidelberg,
       Germany: Springer-Verlag, 2008, pp.~455--470.
\end{enumerate}


\subsection*{Articles in Magazines}
\begin{enumerate}
 \item T. Asami, \underline{J. Kurihara}, D. Kondo, and H. Tode, ``Network operations as an infrastructure for diverse businesses,'' \textit{Journal of Institute of Electronics, Information and Communication Engineers}, vol.~103, no.~2, pp.~155--161, Feb. 2020. [Online]. Available: \url{https://www.journal.ieice.org/bin/pdf_link.php?fname=k103_2_155&lang=E&year=2020} (in Japanese).
 \item \underline{J.~Kurihara}, R.~Matsumoto, and T.~Uyematsu, ``Security of secret-sharing schemes can be characterized by relative parameters of linear codes (Invited paper),'' \textit{IEICE ESS Fundamentals Review}, vol.~9, no.~1, pp.14--23, Jul. 2015. [Online]. Available: \url{https://www.jstage.jst.go.jp/article/essfr/9/1/9_14/_pdf} (in Japanese).
 \item \underline{J.~Kurihara} ``A stream authentication scheme for 1-seg broadcasting,'' \textit{Material Stage}, vol.~7, no.~12, pp.~22--25, Mar. 2008. (in Japanese)
\end{enumerate}

\subsection*{Miscellaneous (Technical papers/talks)}

\begin{enumerate}
 \item H.~Kimura, \underline{J.~Kurihara}, and T.~Tanaka, ``Security Analysis of the Smart Lock Products against the Device Hijacking Attacks,'' to appear in \textit{Proceedings of the Computer Security Symposium 2024 (CSS 2024)}, Kobe, Japan, Oct.~22--25, 2022.
 \item S.~Hashimoto, T.~Tanaka, and \underline{J.~Kurihara}, ``A Monitoring Mechanism to realize the Byzantine Resilience in Federated Learning via Client Tracking,'' to appear in \textit{Proceedings of the Computer Security Symposium 2024 (CSS 2024)}, Kobe, Japan, Oct.~22--25, 2022. (in Japanese)
 \item K.~Kita, \underline{J.~Kurihara}, and T.~Tanaka, ``An Authorization Method of Computing Resources for Handover of Edge Nodes in MEC Environment,'' to appear in in \textit{Proceedings of the 2024 Society Conference of IEICE}, Saitama, Japan, Sep.~10--13, 2024. (in Japanese)
 \item R.~Watanabe, A.~Kubota, \underline{J.~Kurihara}, and K.~Sakurai, ``Extension of Resource Authorization Method with SSI in Edge Computing,'' \textit{CSEC Technical Report}, Chiba, Japan, Mar.~18--19, 2024. (in Japanese)
 \item T.~Kanayama, T.~Tanaka, and \underline{J.~Kurihara}, ``A study on privacy preserving detection system for phishing sites,'' in \textit{Proceedings of the 2024 IEICE General Conference},
       Hiroshima, Japan, Mar.~4--8, 2024. (in Japanese)
 \item I.~Kurihara, \underline{J.~Kurihara}, and T.~Tanaka,
       ``A New Security Measure in Secret Sharing Schemes and Secure Network Coding,''
       in \textit{Proceedings of the 2023 Computer Security Symposium 2023 (CSS2023)},
       Fukuoka, Japan, Oct.~30--Nov.~2, 2023. (in Japanese)
 \item I.~Kurihara, \underline{J.~Kurihara}, and T.~Tanaka,
       ``Individual Insecurity in Ramp Secret Sharing Schemes,''
       in \textit{Proceedings of the 2023 IEICE General Conference},
       Saitama, Japan, Mar.~7--10, 2023. (in Japanese)
 \item R.~Watanabe, A.~Kubota, and \underline{J.~Kurihara}, ``Application of generalized deduplication techniques in edge computing environments'' in \textit{Proceedings of the Computer Security Symposium 2022 (CSS 2022)}, Kumamoto, Japan, Oct.~24--27, 2022.
 \item \underline{J.~Kurihara}, ``Security and Privacy in DNS (Tutorial),'' in \textit{Proceedings of the 2021 Society Conference of IEICE}, Online, Sep.~14--17, 2021.
 \item \underline{J.~Kurihara}, and T. Kubo, ``Mutualized oblivious DNS ($\mu$ODNS): Hiding a tree in the wild forest,’’ \textit{Technical Report of IEICE}, vol.~121, no.~102, NS2021-44, pp.~63--68, Jul. 2021.
 \item \underline{J.~Kurihara}, and T. Kubo, ``Mutualized oblivious DNS ($\mu$ODNS): Hiding a tree in the wild forest,’’ Apr. 2021. [Online]. Available: \url{https://arxiv.org/abs/2104.13785}.
\item R.~Watanabe, A.~Kubota, and \underline{J.~Kurihara}, ``Resource Authorization Patterns on Edge Computing,'' \textit{Technical Report of IEICE}, vol.~120, no.~414, IN2020-68, pp.~85--90, Mar. 2021.
 \item \underline{J.~Kurihara}, T. Nakamura, and R. Watanabe, ``On the Resistance to Byzantine and Unresponsive Servers in Code-based PIR Schemes,'' in \textit{Error-Correcting Codes Workshop (ECCWS) 2020}, Online, Sep.~2--3, 2020. (in Japanese)
 \item D. Kondo, \underline{J.~Kurihara}, H. Tode, and T. Asami, ``Name Prefix Security Applications in NDN,'' in \textit{Proceedings of the 2019 Society Conference of IEICE}, Osaka, Japan, Sep.~10--13, 2019.
 \item \underline{J.~Kurihara}, D. Kondo, H. Tode, and T. Asami, ``Introduction to Name Prefix Security in NDN,'' in \textit{Proceedings of the 2019 Society Conference of IEICE}, Osaka, Japan, Sep.~10--13, 2019.
 \item \underline{J.~Kurihara}, and T.~Kubo ``Formal expression of BBc-1 mechanism and its security analysis,'' Oct.~31, 2017. [Online]. Available: \url{https://beyond-blockchain.org/public/bbc1-analysis.pdf}.
 \item \underline{J.~Kurihara}, ``Current security-related topics and content protection in information-centric networking [Tutorial]'' in \textit{Proceedings of the 2016 Society Conference of IEICE}, Hokkaido, Japan, Sep.~20--23, 2016. (in Japanese)
 \item \underline{J.~Kurihara}, and M.~Mosko, ``Proposed proof of concept contribution by KDDI R\&D Labs., Inc.,'' in \textit{ITU-T Focus Group on IMT-2020}, Seoul, Korea, Mar. 8--11, 2016.
 \item \underline{J.~Kurihara}, ``$1$-to-$n$ matching between interest and content objects for reduction of router workload,'' in \textit{Proceedings of the 94-th IETF Meeting}, IRTF ICNRG, Yokohama, Japan, Nov. 4, 2015.
 \item \underline{J.~Kurihara}, K.~Yokota, K.~Ueda, and A.~Tagami, ``Reduction of router workload by using list-type interests,'' in \textit{Kick-off Workshop of IEICE Technical Committee on Information-Centric Networking}, Tokyo, Japan, Apr. 7, 2015. (in Japanese)
 \item Y.~Yokota, \underline{J.~Kurihara}, A.~Tagami, ``A study of TCP-like congestion control using interest aggregation in content-centric networking,'' in \textit{Technical Report of IEICE. NS}, vol.~114, no.~477, pp.~173--178, Mar. 2015. (in Japanese)
 \item B.~Namsraijav, T.~Asami, Y.~Kawahara, \underline{J.~Kurihara}, K.~Sugiyama, A.~Tagami, T.~Yagyu, and T.~Hasegawa, ``Identity-based aggregate signatures applied to NDN for short message transfers,'' in \textit{Technical Report of IEICE. IN}, vol.~114, no.~478, pp.~319--324, Mar. 2015.
 \item T.~Sunaga, T.~Asami, Y.~Kawahara, K.~Sugiyama, \underline{J.~Kurihara}, A.~Tagami, T.~Yagyu, and T.~Hasegawa, ``Optimization of ICN potential based routing for disasters,'' in \textit{Technical Report of IEICE. IN}, vol.~114, no.~478, pp.~313--318, Mar. 2015. (in Japanese)
 \item \underline{J.~Kurihara}, R.~Matsumoto, and T.~Uyematsu, ``Security of secret-sharing schemes can be characterized by relative parameters of linear codes (Invited talk),'' in \textit{Technical Report of IEICE. IT}, vol.~114, no.~470, pp.~239--246, Feb. 2015. (in Japanese)
 \item \underline{J.~Kurihara}, ``Relative generalized rank weight of linear codes and its applications to network coding (Invited talk),'' in \textit{SITA 2014 workshop on current topics of coding in distributed systems}, Toyama, Japan, Dec. 9--12, 2014. (in Japanese)
 \item \underline{J.~Kurihara}, ``A secret sharing scheme based on linear codes and its security analysis (invited talk)'', in \textit{Workshop on Discrete Mathematics Related to Information Security}, Nagano, Japan, Aug. 2013. (in Japanese)
 \item \underline{J.~Kurihara}, T.~Uyematsu, and R.~Matsumoto,
       ``Secret sharing schemes can be precisely
       characterized by the relative generalized Hamming weight,''
       in \textit{Proceedings of the 2012 IEICE General Conference},
       Okayama, Japan, Mar.~20--23, 2012.
 \item \underline{J.~Kurihara}, ``An XOR-based high-speed secret sharing (Invited talk),'' in \textit{One day workshop on secret sharing and cloud computing, Institute of Mathematics for Industry, Kyushu University}, Fukuoka, Kyushu, Jun. 2011.
 \item \underline{J.~Kurihara}, and T.~Uyematsu, ``Time-varying jamming links for MRD-code-based universal secure error-correcting network codes,'' in \textit{Proceedings of the 2010 Society Conference of IEICE}, Osaka, Japan, Sep.~14--17, 2010. (in Japanese)
 \item \underline{J.~Kurihara}, and T.~Uyematsu, ``Strongly-secure secret
       sharing based on linear codes can be characterized by generalized
       Hamming weight,'' in \textit{Technical Report of IEICE. IT},
       vol.~111, no.~142, pp.~35--40, Jul. 2007.
 \item Y.~Nakano, \underline{J.~Kurihara}, S.~Kiyomoto, and T.~Tanaka, ``A message injection in SCH,'' in \textit{Proceedings of the 2010 IEICE General Conference}, Miyagi, Japan, Mar.~16--19, 2010.

 \item \underline{J.~Kurihara}, T. Uyematsu, S.~Kiyomoto, K.~Fukushima,
       and T.~Tanaka, ``Rediscovery of XOR-based threshold schemes in MDS codes,'' in \textit{Proceedings of the 27th
       Symposium on Cryptography and Information Security (SCIS 2010)},
       Takamatsu, Japan, Jan.~19--22, 2010.
 \item \underline{J.~Kurihara}, T. Uyematsu, S.~Kiyomoto, K.~Fukushima,
       and T.~Tanaka, ``A novel realization of $(k,n)$-threshold schemes
       over binary field extensions,'' in \textit{Proceedings of the 27th
       Symposium on Cryptography and Information Security (SCIS 2010)},
       Takamatsu, Japan, Jan.~19--22, 2010.
 \item Y.~Nakano, \underline{J.~Kurihara}, S.~Kiyomoto, and T.~Tanaka, ``A study on stream-cipher-based hash functions,'' in \textit{Technical Report of IEICE. SITE}, vol.~109, no.~114, pp.~153-159, Jun. 2009.
 \item  \underline{J.~Kurihara}, S.~Kiyomoto, K.~Fukushima, and T.~Tanaka, ``Revocation and addition mechanisms for fast $(k, n)$-threshold schemes'' in \textit{Proceedings of the 2009 IEICE General Conference}, Ehime, Japan, Mar.~17--20, 2009.
 \item \underline{J.~Kurihara}, S.~Kiyomoto, K.~Fukushima, and T.~Tanaka, ``Fast $(k,n)$-threshold schemes for hierarchical access structures'' in \textit{Proceedings of the Computer Security Symposium 2008 (CSS 2008)}, Okinawa, Japan, Oct.~8--10, 2008.
 \item \underline{J.~Kurihara}, S.~Kiyomoto, R.~Watanabe, and T.~Tanaka, ``A stream authentication scheme for 1-seg broadcasting,''in \textit{Proceedings of the 2008 IEICE General Conference}, Fukuoka, Japan, Mar.~18--21, 2008. (in Japanese)
 \item \underline{J.~Kurihara}, S.~Kiyomoto, K.~Fukushima, and T.~Tanaka,
       ``A new $(k,n)$-threshold secret sharing scheme and its
       extension,'' Cryptology ePrint Archive, Report 2008/409,
       2008. [Online]. Available: \url{http://eprint.iacr.org/2008/409}.
 \item \underline{J.~Kurihara}, S.~Kiyomoto, K.~Fukushima, and
       T.~Tanaka, ``An extension of fast threshold schemes using XOR
       operations (2),'' in \textit{Technical Report of IEICE. ISEC},
       vol.~107, no.~209, pp.~9--15, Sep. 2007.
 \item \underline{J.~Kurihara}, S.~Kiyomoto, K.~Fukushima, and
       T.~Tanaka, ``An extension of fast threshold schemes using XOR
       operations (1),'' in \textit{Technical Report of IEICE. ISEC},
       vol.~107, no.~209, pp.~1--8, Sep. 2007.
 \item \underline{J.~Kurihara}, S.~Kiyomoto, K.~Fukushima, and
       T.~Tanaka, ``A fast $(4,n)$-threshold secret sharing scheme using
       exclusive-or operations, and its extension to $(k,n)$-threshold
       schemes,'' in \textit{Technical Report of IEICE. ISEC}, vol.~107, no.~44, pp.~23--30, May
      2007.
 \item \underline{J.~Kurihara}, S.~Kiyomoto, K.~Fukushima, and
       T.~Tanaka, ``The completeness proof of $(3,n)$-threshold secret
       sharing scheme using XOR operations,'' in \textit{Proceedings of the
       24th Symposium on Cryptography and Information Security (SCIS 2007)},
       Nagasaki, Japan, Jan.~23--26, 2007. (in Japanese)
 \item \underline{J.~Kurihara}, S.~Kiyomoto, K.~Fukushima, and
       T.~Tanaka, ``A $(3,n)$-threshold secret sharing scheme using XOR
       operations,'' in \textit{Proceedings of the 24th Symposium on
       Cryptography and Information Security (SCIS 2007)}, Nagasaki,
       Japan, Jan.~23--26, 2007. (in Japanese)
 \item \underline{J.~Kurihara}, T.~Uyematsu, and R.~Matsumoto, ``Efficient nonbinary coding systems approaching the shannon limit by using product accumlate codes,'' in \textit{Technical Report of IEICE. CS}, vol.~105, no.~460, pp.~45--50, Dec. 2004. (in Japanese)
 \item \underline{J.~Kurihara}, and H.~Suzuki ``Software radio receiver utilizing RF filter bank,'' in \textit{Technical Report of IEICE. RCS}, vol.~104, no.~257, pp.~79--84, Aug. 2004. (in Japanese)
\end{enumerate}


\section*{Patents}
48+ patents on distributed storage codes, network coding, secret sharing schemes, stream authentication, information-centric networking, etc.\@ have been filed in Japan. Until Feb. 2019, 27 patents have been accepted in Japan. Some of them have also been filed in US as well and one patent have been accepted.



\section*{Honors and Awards}
\begin{itemize}
\item IEICE Kiyasu Zen'ichi (Best Paper) Award (2014).
\item IEICE Excellent Paper Award (2014).
\item IEICE Academic Encouragement Award of Engineering Sciences Society (2013).
\item Excellent Paper Award in Computer Security Symposium 2008 (2008).
\end{itemize}

\section*{Grants from External Organizations}
\begin{itemize}
 \item JSPS KAKENHI (Grant no. JP20K23329), PI, 2020-2021
 \item JSPS KAKENHI (Grant no. JP21H03442), Co, 2021-2023
 \item University of Hyogo (Special Research Grant (Young Researchers)), PI, 2020/2021
 \item KDDI Research, Inc. (Funded Research Project), PI, 2020/2021
 \item HORIZON2020 (Grant Agreement No. 723014) / NICT (Contract No. 184), 2016--2019
 \item NICT (Contract No. 19103), 2016--2021
\end{itemize}



\section*{Membership}
\begin{itemize}
\item A member of the Institute of Electrical and Electronics Engineers, Inc. (IEEE)
\item A member of the Institute of Electronics, Information and Communication Engineers
(IEICE) of Japan.
\end{itemize}


\section*{Certifications}
\begin{itemize}
\item Network Specialist (Dec. 2017, Information-technology Promotion Agency (IPA), Japan)
\item Registered Information Security Specialist (Jun. 2017, Information-technology Promotion Agency (IPA), Japan)\footnote{National qualification in cybersecurity. Passed the examination but not registered to the Japanese government yet. Registration is possible anytime.}
\item Applied Information Technology Engineer (Dec. 2016, Information-technology Promotion Agency (IPA), Japan)
\end{itemize}

\section*{Lectures}
\begin{itemize}
 \item Security Engineering (2020--2021, Graduate School of U-Hyogo)
 \item Network Security (2020--2021, Graduate School of U-Hyogo)
 \item Information Security (2020--2021, Graduate School of U-Hyogo)
 \item FIDO2 –Modern Authentication– (2020, Zettant)
 \item Introduction to End-to-End Encryption using JavaScript (2019, Zettant)
\end{itemize}

\section*{Professional Services}
\begin{itemize}
 \item Program Committee, System Track, IPSJ CSS 2022, 2023, 2024
 \item Organizing Committee, IPSJ CSS 2020.
 \item Technical Program Committee, DISS Workshop in NDSS 2019.
 \item Technical Program Committee, ACM ICN 2018.
 \item Poster and Demo Program Committee, ACM SIGCOMM 2017.
 \item Organizing Committee, ACM ICN 2016.
 \item Organizing Committee, IEICE SCIS 2010.
 \item Reviewer for
%jounal
\textit{IEEE Transactions on Information Theory},
\textit{IEEE Transactions on Information Forensics and Security},
\textit{IEEE Journal on Selected Areas in Communications},
\textit{IEEE Communications Letters},
\textit{IEICE Transactions on Fundamentals of Electronics, Communications and Computer Sciences},
\textit{IEICE Transactions on Communications},
\textit{IEICE Transactions on Information and Systems},
\textit{IPSJ Journal},
\textit{Advances in Mathematics of Communications},
%conference
\textit{IEEE International Symposium on Information Theory},
\textit{IEEE International Communication Conference},
\textit{IEEE Globecomm},
\textit{IEEE Information Theory Workshop},
\textit{IEEE International Symposium on Network Coding},
\textit{International Symposium on Information Theory and Its Applications},
\textit{ACM International Conference on Information-centric Networking},
%
etc.
\end{itemize}


\bigskip

% Footer
\begin{center}
  \begin{footnotesize}
   % Last updated:
   \today \\
    \href{\footerlink}{\texttt{\footerlink}}
  \end{footnotesize}
\end{center}
